\documentclass{article}
\usepackage{verbatim}
\begin{document}
Now that we have a \verb|rweb.rb| file that does the correct job
of executing the literate Ruby code given, the next step in literate
programming is to actually provide a nice display of the program.

\verb|rweb2text.rb| converts the text and the code of a literate Ruby
program into (hopefully) nicely formatted LaTeX.

The \verb|RWebBeautifier| is the main class.

\begin{verbatim}
 class RWebBeautifier
\end{verbatim}

Now, we copy the regular expressions to parse the literate programs
straight from \verb|rweb|

\begin{verbatim}
# Escapes a whole line if it starts with this regular expression: the
# rest of the line is fed as is to the current output (text or code)
# without interpretation.
ESCAPE = /^\s*@@/

# Inline code
INLINE = /^\s*>+/

# Beginning of a code block
B_o_CODE = /^\s*\\begin\{code\}\s*$/

# End of a code block
E_o_CODE = /^\s*\\end\{code\}\s*$/
\end{verbatim}

Initialization; as I don't provide many hooks, this is rather simple:
\verb|cls| is the document class, \verb|code_env| is the name of
the environment used to display code and \verb|packages|
a list of packages to be included.
\begin{verbatim}
def initialize(cls = 'article', 
               code_env = 'verbatim', 
               packages = ['verbatim'])
  @document_class = cls
  @code_env = code_env
  @packages = packages
end
\end{verbatim}

The \verb|literate_lines| function is a rewrite of
\verb|unliterate_lines| from \verb|rweb.rb| --- a rewrite is indeed
needed as special formatting is required for included code, and
we don't really care about getting only code.

\begin{verbatim}
def literate_lines(lines)
  text = []
  code = []
\end{verbatim}

This time, \verb|code| holds a different meaning: it is the current code
block, not all the code read so far.

\begin{verbatim}
  current = text
  for line in lines
    case line
    when ESCAPE               # Escaping
      current << $'
    when INLINE
      code << $'
    when B_o_CODE
      if current == code
        current << line
      else
        current = code
      end
    when E_o_CODE
      if current == text
        current << line
      else
        current = text
      end
    else
\end{verbatim}
Now come the real difference: if we are in text mode, we need to flush first
the code which hasn't been written yet.
\begin{verbatim}
      if (current == text) and (not code.empty?)
        current << "\\begin{#{@code_env}}\n"
\end{verbatim}
Here, I had first coded using a simple \verb|+=|, but that miserably
fails to work, because after it, \verb|current| is neither \verb|text|
nor \verb|code|, and the code is lost. The solution is
\begin{verbatim}
        current.concat(code)
        current << "\\end{#{@code_env}}\n"
        code.clear
      end
      current << line
    end
  end
  return text
end
\end{verbatim}

Now, a simple function that wraps the appropriate
\verb|literate_lines| call for a file. I find it self-explanatory.
\begin{verbatim}
def literate_file(file)
  output_file = file.sub(/(\.lrb)?$/, '.tex')
  out = File.open(output_file, 'w')
  out.puts "\\documentclass{#{@document_class}}"
  @packages.each do |p|
    out.puts "\\usepackage{#{p}}"
  end
  out.puts "\\begin{document}"
  out.puts(literate_lines(File.open(file).readlines))
  out.puts "\\end{document}"
  out.close
end
\end{verbatim}

The end of the class.

\begin{verbatim}
 end
\end{verbatim}

Now, what is left is some wrapper call; we first create an instance of
\verb|RWebBeautifier|

\begin{verbatim}
 rweb = RWebBeautifier.new
\end{verbatim}

And we use it on all command-line arguments:

\begin{verbatim}
 ARGV.each do |file|
 rweb.literate_file(file)
 end
\end{verbatim}

And that's all ! 
\end{document}
